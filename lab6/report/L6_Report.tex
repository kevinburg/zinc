\documentclass[aps,letterpaper,11pt]{revtex4}
\input kvmacros
\newcommand{\tildechar}{\mbox{\tt\char"7E}}
\newcommand{\ulinechar}{\mbox{\tt\char"5F}}
\newcommand{\percentchar}{\mbox{\tt\char"25}}

\newcommand{\m}[1]{\mathsf{#1}}

\newcommand{\ch}[1]{\mbox{\tt #1}}
\newcommand{\lp}{\mbox{\tt(}}
\newcommand{\rp}{\mbox{\tt)}}
\newcommand{\lb}{\mbox{\tt[}}
\newcommand{\rb}{\mbox{\tt]}}
\newcommand{\eps}{\epsilon}
\newcommand{\arrow}{\longrightarrow}
\newcommand{\reg}[1]{\mathtt{\percentchar #1}}
\newcommand{\stackp}[1]{\begin{array}[b]{l} #1\end{array}}
\newcommand{\stackc}[1]{\begin{array}[t]{l} #1\end{array}}
\newcommand{\live}{\mathsf{live}}
\newcommand{\comp}{\mathrel{\mbox{?}}}


\linespread{1.2}

\usepackage{proof}
\usepackage{graphicx}
\usepackage{float}
\usepackage{verbatim}
\usepackage{amsmath}
\usepackage{amssymb}
\usepackage{fullpage}
\usepackage{listings}
\usepackage{subfig}
\usepackage[usenames,dvipsnames]{color}
\usepackage[pdftex]{hyperref}
\hypersetup{
    colorlinks,
    citecolor=blue,
    filecolor=blue,
    linkcolor=blue,
    urlcolor=blue
}
\definecolor{gray}{gray}{.96}
\newcommand{\labtitle}{Lab 6 - Polymorphism}
\newcommand{\authorname}{Kevin Burg - John Cole}
\newcommand{\ttt}[1]{\texttt{#1}}
\newcommand{\bb}[1]{\textbf{#1}}

\begin{document}

\begin{titlepage}
\begin{center}
{\Large \textsc{\labtitle} \\ \vspace{4pt}} 
\rule[13pt]{\textwidth}{1pt} \\ \vspace{150pt}
{\large By: \authorname \\ \vspace{10pt}
\today}
\end{center}
\end{titlepage}


\section{Introduction}
There are two principle forms of polymorphism: \emph{ad hoc polymorphism} and \emph{parametric polymorphism}.
L5 supports ad hoc polymorphism with its equality \texttt{==} and inequality \texttt{!=} operators. The goal
of our lab is to extend L5 to support parametric polymorphism. We do this by adding language features that
allow for polymorphic datatype definitions and for defining functions which behave uniformly across various
types.

Consider the following program L6 program.

\lstset{language=C, tabsize=3, backgroundcolor=\color{gray}, basicstyle=\small, frame=single,
		commentstyle=\color{BrickRed}, numbers=left}
\lstinputlisting{PolyExample.c}

We say the type \texttt{struct foo} is parameterized by the type \texttt{a}. This is useful because we can 
reuse the struct definition and the \texttt{getSize} function throughout the program on various types.
At line 12, we see an example of instantiating a value of type \texttt{struct foo} where the type
\texttt{int} is substituted for the type parameter \texttt{a}. We enforce that only small types may be
substituted for these type parameters.

Another addition in Lab 6 that we included, was function pointers. While this
seems like a small addition, it allowed us to more effectively use
parametric polymorphism. We could pass function with specific types to a more
general execution function, and reuse the general execution function for
many different specifically typed functions with our polymorphic data.

Our implementation of function pointers only extended to the point of using
function pointers only as expressions or parameters to functions. Function
pointers could not be stored in structs, which is something we wish we could
have implemented if we had the opportunity to do this again.

\newpage
\section{Specification}

We extend the grammar of L4 in the following way.\\

\begin{tabular}{lll}
$\langle\text{identopt}\rangle$ \hspace{.5in}	& $::=$ & $\epsilon~|~\langle\text{ident}\rangle$\\
$\langle\text{sdecl}\rangle$	& $::=$	& 	\textbf{struct ident} $\langle$identopt$\rangle$\textbf{;}\\
$\langle\text{sdefn}\rangle$	& $::=$	& 	\textbf{struct ident} $\langle$identopt$\rangle$ \textbf{\{}
											$\langle\text{field-list}\rangle$ \textbf{\};}\\
$\langle\text{type}\rangle$	& $::=$	& \ldots $|$ \textbf{$<$}$\langle\text{type}\rangle$\textbf{$>$}
\end{tabular}\\

We need to introduce some notation to explain parameterized structure types. We allow a new form of declaration
$a : \text{type}$ in our context. We write the type of a polymorphic structure in the following way

\begin{center}
$s : \forall a_1, \ldots, a_n. s[a_1, \ldots, a_n]$
\end{center}

In the example from the introduction, we would write $\textbf{struct foo} : \forall a. \textbf{struct foo}[a]$.
Now write the judgement which gives a type to instantiated polymorphic structures is given by
\[
\begin{tabular}{c}
  \infer{\textbf{alloc}(s <\tau_1, \ldots, \tau_n>) : \tau[\tau_1, \dots, \tau_n]^*}
        {	
        	{\tau_1~\text{small}, \ldots, \tau_n~\text{small}} \vspace{.1cm}\\
			{s : \forall a_1, \ldots, a_n. \tau[a_1, \ldots, a_n]}
        }
\end{tabular}
\]
This says two important things. First, it says that any type substitution must occur with a small type.
These are \texttt{int}, \texttt{bool}, etc. Because the small types are all concrete types, and all type
parameters must be provided at allocation, an allocation may never return an object of existential type.

Function types may also be parameterized by type variables. For example, from the introduction,
$$\texttt{getSize} : \forall a. (\textbf{struct foo}[a]) -> \texttt{int}$$

Type checking the use of polymorphic functions requires a definition of type variable substitution. Suppose
$\theta$ is our substitution, then the judgement that allows us to type check the use of polymorphic
functions is given by

\[
\begin{tabular}{c}
  \infer{\Gamma \vdash f(e_1, \ldots, e_n) : \theta(\tau)}
        {	
        	{f : \forall a_1, \ldots, a_k. (\tau_1, \ldots, \tau_n) \rightarrow \tau}\\
			{\Gamma \vdash \theta : (a_1, \ldots, a_k)}\\
			{\Gamma \vdash e_1 : \theta(\tau_1), \ldots, e_n : \theta(\tau_n)}
        }
\end{tabular}
\]

The only way that a function may be parameterized by type variables, is if the type variables occur free in
struct types. We have not yet implemented the functionality that would allow us to write a function that
has the following type
$$\texttt{foo} : \forall a. (a~\textbf{list}) -> \texttt{int}$$

This is something we would implement in the future.

We also implemented function pointers which extended the grammar of L4 to
contain the following:\\
\begin{tabular}{ll}
$\langle$param$\rangle$ & $::=$ $\dots|\;\langle$type$\rangle($\ttt{*}
                                 \bb{ident}$)\langle$type-list$\rangle$\\
$\langle$type-list$\rangle$ & $::=$ $\bb{( )} |\;\bb{(}\langle$type$\rangle$
                                     $\langle$type-list-follow$\rangle$\bb{)}\\
$\langle$type-list-follow$\rangle$ & $::=$ $\epsilon\;|\;,\langle$type$\rangle$
                                           $\langle$type-list-follow$\rangle$\\
$\langle$expr$\rangle$ & $::=$ $\dots|\;($\ttt{*}$\langle$\bb{ident}$\rangle)$
                               $\langle$arg-list$\rangle\;|\;($\ttt{\&}
                               \bb{ident}$)\langle$expr$\rangle$\\
\end{tabular}\\
We use the \ttt{\&} symbol as an operator to retrieve the address of a
function, similar to C. We call function pointers, similar to how we call
functions, except we dereference the function pointer first, and then apply
the arguments to the dereferenced function pointer. We can have function
pointers as parameters to functions, and in this case, we require a list of
types of input to the function as part of the parameter to fully know what
type of function with which we are dealing.

The rules for getting the address of a function pointer are similar to those
of dereferencing a pointer, and the rules for applying arguments to a
dereferenced function pointer are the same as those for function calls with
the addition of dereferencing the pointer to begin with.

\newpage
\section{Implementation}
\subsection{Background}
First we will describe the execution of our compiler before going into detail about the parts which were changed
in lab 6. The program is parsed into a form we call PST before being elaborated into a form we call AST. 
Elaboration follows a set of rules for translating sequences of statements into a form that makes type checking
more natural. The elaboration phase also takes care of separating all the information in the PST into groups
such as function definitions, type definitions, and structure definitions. The CheckAST module then reads
through this AST structure to check that it is well-formed according to the statics of our language definition.

Upon success, we send the \emph{PST} to the CodeGen module where the rest of
translation, register allocation, and optimization happens. This means that we
\emph{throw out} the AST after static checks and return to the data structure
we had before elaboration. Obviously, some information is lost here, but in
the earlier labs, it seemed more natural to generate code from the PST than
the AST. Now it means we have to type check expressions \emph{during} code
generation to ensure we do everything correctly. This isn't ideal. If we could 
do it all again, we would sort out this problem with our IR.

\subsection{Parametric Polymorphism}
Parametric polymorphism itself does not require much change to the code
generation part of the compiler. Most of the changes were to the CheckAST
module. One of the changes that needed to be made was determining the types
of struct field accesses when the structs are parameterized by type variables.
From the example in introduction, we needed to determine the type of
\texttt{s->array} on line 14. The way we did this was to extend the datatype
which held the type of structs with a field for map of types that may have
been substituted in at allocation. Then, when the access happens, we see that
\texttt{s->array} has type \texttt{$a$ list}. By looking into themap, we see
that $a$ was substituted with \texttt{int}, making the actual type
\texttt{int list}.

Another problem is checking that function calls to polymorphic functions are
correct. Consider the following function definition.\\

\lstset{language=C, tabsize=3, backgroundcolor=\color{gray}, basicstyle=\small, frame=single,
		commentstyle=\color{BrickRed}, numbers=none}
\lstinputlisting{PolyFunction.c}

We would like to only allow calls to this function where the first and second argument have the same substitution
for $a$. To do this, we build up the substitution from left to right, making sure that any additional 
substitutions are consistent with what we already have.

\subsection{Function Pointers}
Function pointers are extremely useful in polymorphic code because it allows
users to define functions that can take in another function that will process
the polymorphic objects. Take for example the following code:
\lstset{language=C,tabsize=3,backgroundcolor=\color{gray},basicstyle=\small,
        frame=single, commentstyle=\color{BrickRed},numbers=none}
\lstinputlisting{fnptrs.c}

Here we can see that the function \ttt{doInt} takes in two ints and a
function pointer that does some action on integers and return an int. The
\ttt{doInt} function then returns the result of the function pointer. This
allows the user to write simple functions on ints and use them as arguments
to a broader do some action on ints function. It lets programmers reuse
code to a greater extent.

To implement function pointers, we first looked at how \ttt{gcc} compiles
them and set about reverse engineering the functionality. The first thing we
realized is that x86 has the capability to call a function from a register.
The instruction \ttt{call} can take a register argument preceded by an
asterisks. The next thing we saw, was that to load the address into a function
was to use the \ttt{lea} instruction with the \ttt{\%rip} register. The
exact syntax is to use the name of the function as an offset to \ttt{\%rip}
and use the \ttt{leaq} instruction to store the address of the function into
some register you specify as a destination. Shown in the example below
we load the function \ttt{func\_pointer} into the \ttt{\%rax} register and then
call the function by using the \ttt{call} instruction with the \ttt{\%rax}
register preceded by an asterisk.\\
\lstset{language={[x86masm]Assembler},tabsize=3,backgroundcolor=\color{gray},
        basicstyle=\small,frame=single,numbers=none,
        escapeinside={\^*}{*^}}
\begin{lstlisting}
leaq    _func_pointer(%rip),%rax
call    *%rax
\end{lstlisting}
From this we could begin to parse, check and generate code for function
pointers.

Parsing function pointers forced us to change up the parse a bit, to work with
different formats for expressions and parameters to functions. We also had
to parse the \ttt{\&} operator, which we used for obtaining the address of
a function. Expressions now had to include function application, or applying
arguments to a dereferenced function. Parameters now could be parsed like
expressions in the case of function pointers, but had to be formatted in the
correct way otherwise the compiler would fail to parse the source code. The
example above shows the correct way to include a function pointer as a
parameter in the function definition or declaration.

We had to make some modifications to our semantic checker and typechecker.
Parameters were more complicated, but how we represented the type of functions
as a map from input types to output certainly helped simplify things. We could
allow functions as arguments to functions, but only in the form of a pointer
to a function. We dealt with checking expressions with the \ttt{\&} operator,
and calling function pointers as expressions. When calling function pointers
we had to make sure that the arguments to the function pointer matched the
parameters to the function.

Generating code for function pointers wasn't difficult. We added
implementations for the \ttt{\&} operator, and calling function pointers,
which wasn't much different than normal function calls with moving
arguments into the correct registers, and moving the result into the correct
location as well. The only real difference was, as mentioned, the call
instruction with a register as an argument. We also took care of allocating
registers for the function pointer, which acted just like a variable or temp.

Function pointers provided a great benefit to polymorphism, with not too large
of a cost of implementation difficulties. Code reuse is important and function
pointers are another great way to accomplish that.

\newpage
\section{Testing Methodology}
We went about testing in a similar manner to the rest of the labs. We wrote
some tests that would test different parts of what we implemented for Lab 6.
We started with tests for the polymorphism we implemented. We also wrote
some short functions to test that we implemented function pointers correctly.

We also ran the previous labs testing suites on our compiler too, testing to
make sure we didn't accidentally break any of the basic functionality of
\ttt{C0} in the process of implementing polymorphism or function pointers.\\

\newpage
\section{Analysis}
We wanted to extend our compiler in a direction in which was relevant to
popular language aspects that was not already a part of \ttt{C0}. We decided
on polymorphism because we felt it increased the functionality of the language
the largest without changing the semantics too much.

We implemented some of the most important features of polymorphism, as we
thought. Given more time, we would have liked to implement more general
aspects of polymorphism, as mentioned in the introduction. We would have also
liked to extend the use of function pointers to be included in structs, so
that a polymorphic struct could have its own functions which could be used
more effectively for data abstraction.

We made some questionable design decisions in the beginning that came back
to haunt us, as we mentioned with the intermediate representation. We also
had trouble in the beginning and through out writing clean Haskell code, since
neither of us knew Haskell previously. We felt like we were trying to write
ML code in Haskell instead of using the powerful features of the Haskell
language. We eventually realized some of the features and started to use them
more effectively, but we still had large amounts of messy code. Given the
opportunity to do this again, we wouldn't change our choice to use Haskell,
but should have tried to understand the effective techniques of coding in
Haskell, instead of trying to write ML like code in Haskell.

We made some good decisions, especially with regard to what optimizations we
implemented in Lab 5. We implemented a number of optimizations that
increased the efficiency of our generated code incredibly, and given the
chance, or more time, would have implemented many more. We both enjoyed
writing this compiler for many reasons both different and the same. Although
we spent many nights debugging hard to find errors, and being frustrated at
the tests other students wrote, we thoroughly enjoyed being in this class and
learning about compiler design and implementing a compiler.
\end{document}
