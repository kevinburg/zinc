\documentclass[aps,letterpaper,11pt]{revtex4}
\input kvmacros
\newcommand{\tildechar}{\mbox{\tt\char"7E}}
\newcommand{\ulinechar}{\mbox{\tt\char"5F}}
\newcommand{\percentchar}{\mbox{\tt\char"25}}

\newcommand{\m}[1]{\mathsf{#1}}

\newcommand{\ch}[1]{\mbox{\tt #1}}
\newcommand{\lp}{\mbox{\tt(}}
\newcommand{\rp}{\mbox{\tt)}}
\newcommand{\lb}{\mbox{\tt[}}
\newcommand{\rb}{\mbox{\tt]}}
\newcommand{\eps}{\epsilon}
\newcommand{\arrow}{\longrightarrow}
\newcommand{\reg}[1]{\mathtt{\percentchar #1}}
\newcommand{\stackp}[1]{\begin{array}[b]{l} #1\end{array}}
\newcommand{\stackc}[1]{\begin{array}[t]{l} #1\end{array}}
\newcommand{\live}{\mathsf{live}}
\newcommand{\comp}{\mathrel{\mbox{?}}}


\linespread{1.2}

\usepackage{proof}
\usepackage{graphicx}
\usepackage{float}
\usepackage{verbatim}
\usepackage{amsmath}
\usepackage{amssymb}
\usepackage{fullpage}
\usepackage{listings}
\usepackage{subfig}
\usepackage[usenames,dvipsnames]{color}
\usepackage[pdftex]{hyperref}
\hypersetup{
    colorlinks,
    citecolor=blue,
    filecolor=blue,
    linkcolor=blue,
    urlcolor=blue
}
\definecolor{gray}{gray}{.96}
\newcommand{\labtitle}{Lab 6 - Polymorphism}
\newcommand{\authorname}{Kevin Burg - John Cole}

\begin{document}

\begin{titlepage}
\begin{center}
{\Large \textsc{\labtitle} \\ \vspace{4pt}} 
\rule[13pt]{\textwidth}{1pt} \\ \vspace{150pt}
{\large By: \authorname \\ \vspace{10pt}
\today}
\end{center}
\end{titlepage}


\section{Introduction}
There are two principle forms of polymorphism: \emph{ad hoc polymorphism} and \emph{parametric polymorphism}.
L5 supports ad hoc polymorphism with its equality \texttt{==} and inequality \texttt{!=} operators. The goal
of our lab is to extend L5 to support parametric polymorphism. We do this by adding language features that
allow for polymorphic datatype definitions and for defining functions which behave uniformly across various
types.

Consider the following program L6 program.

\lstset{language=C, tabsize=3, backgroundcolor=\color{gray}, basicstyle=\small, frame=single,
		commentstyle=\color{BrickRed}, numbers=left}
\lstinputlisting{PolyExample.c}

We say the type \texttt{struct foo} is parameterized by the type \texttt{a}. This is useful because we can 
reuse the struct definition and the \texttt{getSize} function throughout the program on various types.
At line 12, we see an example of instantiating a value of type \texttt{struct foo} where the type
\texttt{int} is substituted for the type parameter \texttt{a}. We enforce that only small types may be
substituted for these type parameters.


\newpage
\section{Specification}

We extend the grammar of L4 in the following way.\\

\begin{tabular}{lll}
$\langle\text{identopt}\rangle$ \hspace{.5in}	& $::=$ & $\epsilon~|~\langle\text{ident}\rangle$\\
$\langle\text{sdecl}\rangle$	& $::=$	& 	\textbf{struct ident} $\langle$identopt$\rangle$\textbf{;}\\
$\langle\text{sdefn}\rangle$	& $::=$	& 	\textbf{struct ident} $\langle$identopt$\rangle$ \textbf{\{}
											$\langle\text{field-list}\rangle$ \textbf{\};}\\
$\langle\text{type}\rangle$	& $::=$	& \ldots $|$ \textbf{$<$}$\langle\text{type}\rangle$\textbf{$>$}
\end{tabular}\\

We need to introduce some notation to explain parameterized structure types. We allow a new form of declaration
$a : \text{type}$ in our context. We write the type of a polymorphic structure in the following way

\begin{center}
$s : \forall a. s[a]$
\end{center}

In the example from the introduction, we would write $\textbf{struct foo} : \forall a. \textbf{struct foo}[a]$.
Now write the judgement which gives a type to instantiated polymorphic structures is given by
\[
\begin{tabular}{c}
  \infer{\textbf{alloc}(s <\tau'>) : \tau[\tau']^*}
        {	
        	{\tau'~\text{small}} \vspace{.1cm}\\
			{s : \forall a. \tau[a]}
        }
\end{tabular}
\]
This says two important things. First, it says that any type substitution must occur with a small type.
These are \texttt{int}, \texttt{bool}, etc. Because the small types are all concrete types, and structs are
never parameterized by more than one type variable, an allocation may never return an object of
existential type.

Function types may also be parameterized by type variables. For example, from the introduction,
$$\texttt{getSize} : \forall a. (\textbf{struct foo}[a]) -> \texttt{int}$$

Type checking the use of polymorphic functions requires a definition of type variable substitution. Suppose
$\theta$ is our substitution, then the judgement that allows us to type check the use of polymorphic
functions is given by

\[
\begin{tabular}{c}
  \infer{\Gamma \vdash f(e_1, \ldots, e_n) : \theta(\tau)}
        {	
        	{f : \forall a_1, \ldots, a_k. (\tau_1, \ldots, \tau_n) \rightarrow \tau}\\
			{\Gamma \vdash \theta : (a_1, \ldots, a_k)}\\
			{\Gamma \vdash e_1 : \theta(\tau_1), \ldots, e_n : \theta(\tau_n)}
        }
\end{tabular}
\]

\newpage
\section{Implementation}

\newpage
\section{Testing Methodology}

\newpage
\section{Analysis}


\end{document}


%%%%%%%%%%%%%%%%
COMMON COMMANDS:
%%%%%%%%%%%%%%%%
% IMAGES
\begin{figure}[H]
   \begin{center}
      \includegraphics[width=0.6\textwidth]{RTL_SCHEM.png}
   \end{center}
\caption{A screenshot of the RTL Schematics produced from the Verilog code.}
\label{RTL}
\end{figure}

% SUBFIGURES IMAGES
\begin{figure}[H]
  \centering
  \subfloat[LED4 Period]{\label{fig:Per4}\includegraphics[width=0.4\textwidth]{period_led4.png}} \\                
  \subfloat[LED5 Period]{\label{fig:Per5}\includegraphics[width=0.4\textwidth]{period_led5.png}}
  \subfloat[LED6 Period]{\label{fig:Per6}\includegraphics[width=0.4\textwidth]{period_led6.png}}
  \caption{Period of LED blink rate captured by osciliscope.}
  \label{fig:oscil}
\end{figure}

% INSERT SOURCE CODE
\lstset{language=Verilog, tabsize=3, backgroundcolor=\color{mygrey}, basicstyle=\small, commentstyle=\color{BrickRed}}
\lstinputlisting{MODULE.v}

% TEXT TABLE
\begin{table}
\begin{center}
\begin{tabular}{|l|c|c|l|}
	x & x & x & x \\ \hline
	x & x & x & x \\
	x & x & x & x \\ \hline
\end{tabular}
\caption{Caption}
\label{label}
\end{center}
\end{table}

% MATHMATICAL ENVIRONMENT
$ 8 = 2 \times 4 $

% CENTERED FORMULA
\[  \]

% NUMBERED EQUATION
\begin{equation}
	
\end{equation}

% ARRAY OF EQUATIONS (The splat supresses the numbering)
\begin{align*}
	
\end{align*}

% NUMBERED ARRAY OF EQUATIONS
\begin{align}
	
\end{align}

% ACCENTS
\dot{x} % dot
\ddot{x} % double dot
\bar{x} % bar
\tilde{x} % tilde
\vec{x} % vector
\hat{x} % hat
\acute{x} % acute
\grave{x} % grave
\breve{x} % breve
\check{x} % dot (cowboy hat)

% FONTS
\mathrm{text} % roman
\mathsf{text} % sans serif
\mathtt{text} % Typewriter
\mathbb{text} % Blackboard bold
\mathcal{text} % Caligraphy
\mathfrak{text} % Fraktur

\textbf{text} % bold
\textit{text} % italic
\textsl{text} % slanted
\textsc{text} % small caps
\texttt{text} % typewriter
\underline{text} % underline
\emph{text} % emphasized

\begin{tiny}text\end{tiny} % Tiny
\begin{scriptsize}text\end{scriptsize} % Script Size
\begin{footnotesize}text\end{footnotesize} % Footnote Size
\begin{small}text\end{small} % Small
\begin{normalsize}text\end{normalsize} % Normal Size
\begin{large}text\end{large} % Large
\begin{Large}text\end{Large} % Larger
\begin{LARGE}text\end{LARGE} % Very Large
\begin{huge}text\end{huge}   % Huge
\begin{Huge}text\end{Huge}   % Very Huge


% GENERATE TABLE OF CONTENTS AND/OR TABLE OF FIGURES
% These seem to have some issues with the "revtex4" document class.  To use, change
% the very first line of this document to "article" like this:
% \documentclass[aps,letterpaper,10pt]{article}
\tableofcontents
\listoffigures
\listoftables

% INCLUDE A HYPERLINK OR URL
\url{http://www.derekhildreth.com}
\href{http://www.derekhildreth.com}{Derek Hildreth's Website}

% FOR MORE, REFER TO THE "LINUX CHEAT SHEET.PDF" FILE INCLUDED!
