\documentclass[aps,letterpaper,11pt]{revtex4}
\input kvmacros

\linespread{1.2}


\usepackage{graphicx}
\usepackage{float}
\usepackage{verbatim}
\usepackage{amsmath}
\usepackage{amssymb}
\usepackage{fullpage}
\usepackage{listings}
\usepackage{subfig}
\usepackage[usenames,dvipsnames]{color}
\usepackage[pdftex]{hyperref}
\hypersetup{
    colorlinks,
    citecolor=blue,
    filecolor=blue,
    linkcolor=blue,
    urlcolor=blue
}

\newcommand{\labtitle}{Lab 5 - Optimizations}
\newcommand{\authorname}{Kevin Burg - John Cole}

\begin{document}  % START THE DOCUMENT!

\begin{titlepage}
\begin{center}
{\Large \textsc{\labtitle} \\ \vspace{4pt}} 
\rule[13pt]{\textwidth}{1pt} \\ \vspace{150pt}
{\large By: \authorname \\ \vspace{10pt}
\today}
\end{center}
\end{titlepage}


\section{Unsafe Mode}

The following list describes all safety checks that are performed in \texttt{safe} mode and not performed in
\texttt{unsafe} mode.

\begin{itemize}
\item Array bounds checking during write and read.
\item Null checking before struct field access. For example, before evaluating \texttt{s->f}, a null check
is performed on \texttt{s}.
\item Check that shift value falls in the range $[0, 31]$ for left and right shifts.
\item Check requested array size non-negative before \texttt{alloc\_array}.
\end{itemize}

The difference between safe and unsafe mode is most noticeable in tests that make frequent memory accesses.
For example, the execution time of \texttt{mmult2.l4} dropped by almost $60\%$ when compiled with the
\texttt{--unsafe} flag. We attribute this to the number of conditional jumps that can be removed from the
code when bounds checking is ignored. Using naive safety checks, an array access inside a loop results in
bounds check every time the loop goes around. There may be ways to compile safely and decrease execution
time by hoisting some of these checks outside of the loop. \texttt{shifts.l4} ran $85\%$ percent faster.
We attribute this to the removal of conditional branching from every arithmetic shift.

\section{Instruction Selection}
Before optimizations, all conditional branches were encoded in a piece of abstract assembly that looks like
\texttt{ifz v l}, which says to check the value of \texttt{v} and jump to \texttt{l} if it is zero. This makes
the translation for a statement like the following rather bulky.

\begin{verbatim}
if (x > y) {
  return 1;
} else {
  return 0;
}
\end{verbatim}

First, the expression \texttt{x>y} is computed down to a value $b$ of zero or one, using a comparison and a set
instruction. Then, $b$ is compared to zero. There is room for optimization here, because our use of set 
instructions and the second comparison is really just a rewrite of the built-in x86 conditional jumps. Our
optimization introduces a new abstract assembly construct \texttt{comp v1 v2 op l}, which says to compare
\texttt{v1} to \texttt{v2} using the binary comparison operator \texttt{op} and jump to \texttt{l} if the result
of the comparison is false. This allows the above example to be translated to a single \texttt{cmpq/cmpl} 
instruction.

The effect of this optimization is most noticeable in programs that make use of loops that go around a
large number of times. The exit conditions are checked much more efficiently using this improved translation.
This optimization works well with copy propagation to make conditional branches really fast. In some cases,
simple branches were taken from 10 lines of assembly down to 2.

 
\section{Constant Propagation and Folding}

The constant propagation optimization is good for reducing some unsimplified arithmetic expressions in the
test cases. Also, having constant and copy propagation made temporary variable allocation in code generation
much simpler. It allows us to relax the way we assign values to temporary variables during code generation 
with the knowledge that the optimization will take care of extraneous moves.

Constant folding interacts very well with copy propagation. It is common for optimizations made in the 
copy propagation pass to open up opportunities for more content propagation and folding. In order to prevent 
passing over the code multiple times, we implemented an optimization pass that applies constant and copy
propagation all at once. This optimization pass will create lots of dead code. However, since the code was
in SSA form, we were able to determine if a line will become dead immediately after the optimization and
remove it as we go. This made our optimization pass more complicated, but allowed us to skip writing
a dead code eliminator.

Certain constant propagations allowed for more constant folds to happen. This is especially noticeable in the
case of computing array offsets. A translation for accessing \texttt{A[0]} becomes much simpler when we took 
into account that $a+0 = a$ for all addresses $a$.

\section{Eliminating Register Moves}

We noticed two patterns in our assembly that could be optimized, so two peephole optimizations were implemented
directly on the x86 assembly. The first was to remove jumps that jump to the next line in the file. A jump
looking like the following is removed.

\begin{verbatim}
    jump l
l:
\end{verbatim}

We also noticed patterns of moves that look like the following.

\begin{verbatim}
movq a x
movq b x
\end{verbatim}

These two lines can be optimized to \texttt{movq b x} as long as the first move is guaranteed to have no effect
on the value of $b$ (this is not always the case).

We chose to implement copy propagation instead of register coalescing as we feel these two optimizations have
the same effect on our generated code and running both would be redundant. Copy propagation along with finding
the SSA-minimal form of the program greatly reduced the number of move instructions in our files.

\section{Common Subexpression Elimination}
We implemented a rather conservative version of common subexpression elimination. The only common computations
we eliminate are multiplications which don't involve memory references. This optimization has the most noticeable
effect on the \texttt{qsort.l4} test. The swap function involves writing to a memory block immediately after
reading from the same block. The multiplication to compute the array access offset is only done once with this
optimization. We certainly could do more to reduce redundant memory loads, but this does something.

Applying this optimization introduces more opportunities for copy propagation. Our solution to this was to
run copy propagation again after this step.

\section{Inlining}
We developed a heuristic to inline functions so we can eliminate some of the
overhead required for function calls. Our heuristic deals with how many
statements there are in the function to be inlined. Instead of just counting
the total number of statements in the function we classify the statements
giving certain weights to each type of statement. For example, simple
assignment statements would be given a weight less than say a for loop.
This gives a good estimate on how complex a function is, and how much
benefit we gain by inlining the function instead of just calling it as it
is.

We deal with conflicts to variable names by adding the function name to the
variable before inlining functions, thus ensuring that we have unique variable
names. We also exclude recursive functions from being inlined, because
just by unrolling the recursion by one iteration adds only a small benefit
to the overall cost.

The inlining optimization works particularly well with other optimizations that
reduce the number of register conflicts and moves, such as copy propagation. 
When those optimizations are implemented along with inlining a certain function,
it may be the case that a function no longer needs to use callee-save registers,
reducing the number of register moves.

\end{document}


%%%%%%%%%%%%%%%%
COMMON COMMANDS:
%%%%%%%%%%%%%%%%
% IMAGES
\begin{figure}[H]
   \begin{center}
      \includegraphics[width=0.6\textwidth]{RTL_SCHEM.png}
   \end{center}
\caption{A screenshot of the RTL Schematics produced from the Verilog code.}
\label{RTL}
\end{figure}

% SUBFIGURES IMAGES
\begin{figure}[H]
  \centering
  \subfloat[LED4 Period]{\label{fig:Per4}\includegraphics[width=0.4\textwidth]{period_led4.png}} \\                
  \subfloat[LED5 Period]{\label{fig:Per5}\includegraphics[width=0.4\textwidth]{period_led5.png}}
  \subfloat[LED6 Period]{\label{fig:Per6}\includegraphics[width=0.4\textwidth]{period_led6.png}}
  \caption{Period of LED blink rate captured by osciliscope.}
  \label{fig:oscil}
\end{figure}

% INSERT SOURCE CODE
\lstset{language=Verilog, tabsize=3, backgroundcolor=\color{mygrey}, basicstyle=\small, commentstyle=\color{BrickRed}}
\lstinputlisting{MODULE.v}

% TEXT TABLE
\begin{table}
\begin{center}
\begin{tabular}{|l|c|c|l|}
	x & x & x & x \\ \hline
	x & x & x & x \\
	x & x & x & x \\ \hline
\end{tabular}
\caption{Caption}
\label{label}
\end{center}
\end{table}

% MATHMATICAL ENVIRONMENT
$ 8 = 2 \times 4 $

% CENTERED FORMULA
\[  \]

% NUMBERED EQUATION
\begin{equation}
	
\end{equation}

% ARRAY OF EQUATIONS (The splat supresses the numbering)
\begin{align*}
	
\end{align*}

% NUMBERED ARRAY OF EQUATIONS
\begin{align}
	
\end{align}

% ACCENTS
\dot{x} % dot
\ddot{x} % double dot
\bar{x} % bar
\tilde{x} % tilde
\vec{x} % vector
\hat{x} % hat
\acute{x} % acute
\grave{x} % grave
\breve{x} % breve
\check{x} % dot (cowboy hat)

% FONTS
\mathrm{text} % roman
\mathsf{text} % sans serif
\mathtt{text} % Typewriter
\mathbb{text} % Blackboard bold
\mathcal{text} % Caligraphy
\mathfrak{text} % Fraktur

\textbf{text} % bold
\textit{text} % italic
\textsl{text} % slanted
\textsc{text} % small caps
\texttt{text} % typewriter
\underline{text} % underline
\emph{text} % emphasized

\begin{tiny}text\end{tiny} % Tiny
\begin{scriptsize}text\end{scriptsize} % Script Size
\begin{footnotesize}text\end{footnotesize} % Footnote Size
\begin{small}text\end{small} % Small
\begin{normalsize}text\end{normalsize} % Normal Size
\begin{large}text\end{large} % Large
\begin{Large}text\end{Large} % Larger
\begin{LARGE}text\end{LARGE} % Very Large
\begin{huge}text\end{huge}   % Huge
\begin{Huge}text\end{Huge}   % Very Huge


% GENERATE TABLE OF CONTENTS AND/OR TABLE OF FIGURES
% These seem to have some issues with the "revtex4" document class.  To use, change
% the very first line of this document to "article" like this:
% \documentclass[aps,letterpaper,10pt]{article}
\tableofcontents
\listoffigures
\listoftables

% INCLUDE A HYPERLINK OR URL
\url{http://www.derekhildreth.com}
\href{http://www.derekhildreth.com}{Derek Hildreth's Website}

% FOR MORE, REFER TO THE "LINUX CHEAT SHEET.PDF" FILE INCLUDED!
